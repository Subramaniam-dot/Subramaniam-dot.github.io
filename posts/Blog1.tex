% *** Authors should verify (and, if needed, correct) their LaTeX system  ***
% *** with the testflow diagnostic prior to trusting their LaTeX platform ***
% *** with production work. IEEE's font choices can trigger bugs that do  ***
% *** not appear when using other class files.                            ***
% The testflow support page is at:
% http://www.michaelshell.org/tex/testflow/


%%*************************************************************************
%% Legal Notice:
%% This code is offered as-is without any warranty either expressed or
%% implied; without even the implied warranty of MERCHANTABILITY or
%% FITNESS FOR A PARTICULAR PURPOSE!
%% User assumes all risk.
%% In no event shall IEEE or any contributor to this code be liable for
%% any damages or losses, including, but not limited to, incidental,
%% consequential, or any other damages, resulting from the use or misuse
%% of any information contained here.
%%
%% All comments are the opinions of their respective authors and are not
%% necessarily endorsed by the IEEE.
%%
%% This work is distributed under the LaTeX Project Public License (LPPL)
%% ( http://www.latex-project.org/ ) version 1.3, and may be freely used,
%% distributed and modified. A copy of the LPPL, version 1.3, is included
%% in the base LaTeX documentation of all distributions of LaTeX released
%% 2003/12/01 or later.
%% Retain all contribution notices and credits.
%% ** Modified files should be clearly indicated as such, including  **
%% ** renaming them and changing author support contact information. **
%%
%% File list of work: IEEEtran.cls, New_IEEEtran_how-to.pdf, bare_jrnl_new_sample4.tex,
%%*************************************************************************
\PassOptionsToPackage{unicode}{hyperref}
\PassOptionsToPackage{hyphens}{url}
\PassOptionsToPackage{dvipsnames,svgnames,x11names}{xcolor}
% Note that the a4paper option is mainly intended so that authors in
% countries using A4 can easily print to A4 and see how their papers will
% look in print - the typesetting of the document will not typically be
% affected with changes in paper size (but the bottom and side margins will).
% Use the testflow package mentioned above to verify correct handling of
% both paper sizes by the user's LaTeX system.
%
% Also note that the "draftcls" or "draftclsnofoot", not "draft", option
% should be used if it is desired that the figures are to be displayed in
% draft mode.
%
\documentclass[
  journal,
]{IEEEtran}%
% If IEEEtran.cls has not been installed into the LaTeX system files,
% manually specify the path to it like:
% \documentclass[journal]{../sty/IEEEtran}
\usepackage[cmex10]{amsmath}
\usepackage{amssymb}
\usepackage{iftex}
\ifPDFTeX
  \usepackage[T1]{fontenc}
  \usepackage[utf8]{inputenc}
  \usepackage{textcomp} % provide euro and other symbols
\else % if luatex or xetex
  \usepackage{unicode-math} % this also loads fontspec
  \defaultfontfeatures{Scale=MatchLowercase}
  \defaultfontfeatures[\rmfamily]{Ligatures=TeX,Scale=1}
\fi
%\usepackage{lmodern}
\ifPDFTeX\else
\fi
% Use upquote if available, for straight quotes in verbatim environments
\IfFileExists{upquote.sty}{\usepackage{upquote}}{}
\IfFileExists{microtype.sty}{% use microtype if available
  \usepackage[]{microtype}
  \UseMicrotypeSet[protrusion]{basicmath} % disable protrusion for tt fonts
}{}
\makeatletter
\parindent    1.0em
\ifCLASSOPTIONcompsoc
  \parindent    1.5em
\fi
\makeatother
\usepackage{xcolor}
\setlength{\emergencystretch}{3em} % prevent overfull lines

\setcounter{secnumdepth}{5}
% Make \paragraph and \subparagraph free-standing
\ifx\paragraph\undefined\else
  \let\oldparagraph\paragraph
  \renewcommand{\paragraph}[1]{\oldparagraph{#1}\mbox{}}
\fi
\ifx\subparagraph\undefined\else
  \let\oldsubparagraph\subparagraph
  \renewcommand{\subparagraph}[1]{\oldsubparagraph{#1}\mbox{}}
\fi


\providecommand{\tightlist}{%
  \setlength{\itemsep}{0pt}\setlength{\parskip}{0pt}}\usepackage{longtable,booktabs,array}
\usepackage{calc} % for calculating minipage widths
% Correct order of tables after \paragraph or \subparagraph
\usepackage{etoolbox}
\makeatletter
\patchcmd\longtable{\par}{\if@noskipsec\mbox{}\fi\par}{}{}
\makeatother
% Allow footnotes in longtable head/foot
\IfFileExists{footnotehyper.sty}{\usepackage{footnotehyper}}{\usepackage{footnote}}
\makesavenoteenv{longtable}
\usepackage{graphicx}
\makeatletter
\def\maxwidth{\ifdim\Gin@nat@width>\linewidth\linewidth\else\Gin@nat@width\fi}
\def\maxheight{\ifdim\Gin@nat@height>\textheight\textheight\else\Gin@nat@height\fi}
\makeatother
% Scale images if necessary, so that they will not overflow the page
% margins by default, and it is still possible to overwrite the defaults
% using explicit options in \includegraphics[width, height, ...]{}
\setkeys{Gin}{width=\maxwidth,height=\maxheight,keepaspectratio}
% Set default figure placement to htbp
\makeatletter
\def\fps@figure{htbp}
\makeatother

\usepackage{physics}
\usepackage[version=3]{mhchem}
\usepackage{orcidlink}
\usepackage{float}
\floatplacement{table}{htb}
\makeatletter
\@ifpackageloaded{caption}{}{\usepackage{caption}}
\AtBeginDocument{%
\ifdefined\contentsname
  \renewcommand*\contentsname{Table of contents}
\else
  \newcommand\contentsname{Table of contents}
\fi
\ifdefined\listfigurename
  \renewcommand*\listfigurename{List of Figures}
\else
  \newcommand\listfigurename{List of Figures}
\fi
\ifdefined\listtablename
  \renewcommand*\listtablename{List of Tables}
\else
  \newcommand\listtablename{List of Tables}
\fi
\ifdefined\figurename
  \renewcommand*\figurename{Fig.}
\else
  \newcommand\figurename{Fig.}
\fi
\ifdefined\tablename
  \renewcommand*\tablename{Table}
\else
  \newcommand\tablename{Table}
\fi
}
\@ifpackageloaded{float}{}{\usepackage{float}}
\floatstyle{ruled}
\@ifundefined{c@chapter}{\newfloat{codelisting}{h}{lop}}{\newfloat{codelisting}{h}{lop}[chapter]}
\floatname{codelisting}{Listing}
\newcommand*\listoflistings{\listof{codelisting}{List of Listings}}
\makeatother
\makeatletter
\makeatother
\makeatletter
\@ifpackageloaded{caption}{}{\usepackage{caption}}
\@ifpackageloaded{subcaption}{}{\usepackage{subcaption}}
\makeatother
\usepackage[skip=2pt,font=footnotesize]{caption}
%\captionsetup{format=myformat}
\ifLuaTeX
  \usepackage{selnolig}  % disable illegal ligatures
\fi
\IfFileExists{bookmark.sty}{\usepackage{bookmark}}{\usepackage{hyperref}}
\IfFileExists{xurl.sty}{\usepackage{xurl}}{} % add URL line breaks if available
\urlstyle{same} % disable monospaced font for URLs
\hypersetup{
  pdftitle={The Democratization of AI: How GPUs and Parallel Computing Are Opening Doors for Everyone},
  colorlinks=true,
  linkcolor={blue},
  filecolor={Maroon},
  citecolor={Blue},
  urlcolor={Blue},
  pdfcreator={LaTeX via pandoc}}

% *** Do not adjust lengths that control margins, column widths, etc. ***
% *** Do not use packages that alter fonts (such as pslatex).         ***
% There should be no need to do such things with IEEEtran.cls V1.6 and later.
% (Unless specifically asked to do so by the journal or conference you plan
% to submit to, of course. )


% correct bad hyphenation here
\hyphenation{op-tical net-works semi-conduc-tor}

%
% paper title
% can use linebreaks \\ within to get better formatting as desired
% Do not put math or special symbols in the title.
% paper title
% can use linebreaks \\ within to get better formatting as desired
% Do not put math or special symbols in the title.
\title{The Democratization of AI: How GPUs and Parallel Computing Are
Opening Doors for Everyone}

\author{

}
\begin{document}

% The paper headers

% use for special paper notices

% make the title area
\maketitle

% As a general rule, do not put math, special symbols or citations
% in the abstract or keywords.
% Note that keywords are not normally used for peerreview papers.

% For peer review papers, you can put extra information on the cover
% page as needed:
% \ifCLASSOPTIONpeerreview
% \begin{center} \bfseries EDICS Category: 3-BBND \end{center}
% \fi
%
% For peerreview papers, this IEEEtran command inserts a page break and
% creates the second title. It will be ignored for other modes.
% \IEEEpeerreviewmaketitle


The landscape of artificial intelligence (AI) development is undergoing
a profound transformation, driven by advancements in GPUs, parallel
computing, and the increasing accessibility of AI technologies. This
evolution is not just technical but also deeply democratic, lowering the
barriers to entry for AI research and development. Let's delve into how
these changes are reshaping the field of AI.

\section{The Unseen Powerhouses: GPUs and Parallel
Computing}\label{the-unseen-powerhouses-gpus-and-parallel-computing}

Historically, the realm of AI was dominated by large research
institutions and corporations with access to significant computational
resources. This began to change with the advent of General-Purpose
Graphics Processing Units (GPUs). Initially designed to accelerate
graphics rendering, GPUs have emerged as pivotal in AI development.
Their ability to perform massive data parallel computing tasks
efficiently makes them ideal for the complex calculations required in
AI, particularly in neural network training.

Research by Lin et al.~{[}1{]} highlighted GPUs' cost-effectiveness and
superior performance, achieving speeds up to 14.74 times faster than
traditional CPU implementations for exact pattern matching. This
capability has revolutionized the way AI models are trained, allowing
for more sophisticated and accurate systems.

Moreover, the computing research community, as noted by Keckler et
al.~{[}2{]}, faces the challenge of scaling in single-chip parallel
computing systems. The exploration into heterogeneous high-performance
computing systems, incorporating GPUs, is a testament to the ongoing
efforts to address these computational demands.

\section{Making AI Accessible: The Role of
Platforms}\label{making-ai-accessible-the-role-of-platforms}

The democratization of AI has been significantly propelled by platforms
offering access to powerful computational resources. Google Colab stands
out as a beacon of accessibility, providing a free and fully configured
runtime for deep learning applications. This platform reduces execution
times dramatically by utilizing GPU parallelism, a game-changer for
processing large-scale data across various research domains.

Shariar and Hasan's work {[}3{]} on ``GPU Accelerated Indexing for High
Order Tensors in Google Colab'' exemplifies the practical benefits of
this approach. By implementing an Index Partitioning Algorithm (IPA) and
a Scalable Tensor Structure (STS), they demonstrated how to enhance
tensor indexing and data processing efficiency on Google Colab,
leveraging GPU parallelism for better performance and load balancing.

\section{Education and AI: Leveling the Playing
Field}\label{education-and-ai-leveling-the-playing-field}

While the scientific literature may not explicitly address platforms
like Fast.ai, the trend is clear: the movement towards making computing
resources and AI education more accessible is unmistakable. These
platforms play a crucial role in democratizing AI by leveling the
playing field, allowing researchers, developers, and enthusiasts from
all over the world to participate in AI development.

The implications of this democratization are profound. By making
high-performance computing resources and educational tools widely
available, we're not only accelerating the pace of AI innovation but
also ensuring that this innovation is inclusive, drawing from a diverse
pool of talent and perspectives.

\section{Conclusion: A Future Powered by Accessible
AI}\label{conclusion-a-future-powered-by-accessible-ai}

The integration of GPUs and parallel computing into AI development has
significantly enhanced performance and efficiency, marking a new era of
innovation. Platforms like Google Colab and educational initiatives are
pivotal in making AI technologies more accessible, fostering a more
inclusive and democratic AI research and development landscape.

As we move forward, the continued expansion of access to AI tools and
resources promises to catalyze further innovation. By democratizing the
field, we ensure that the future of AI is shaped by a broad and diverse
community of thinkers, making the technologies we develop more robust,
ethical, and reflective of our collective needs.

\section{References}\label{references}

{[}1{]} H. Lin, P. Balaji, R. Poole, C. Sosa, X. Ma, and W. -c.~Feng,
``Massively parallel genomic sequence search on the Blue Gene/P
architecture,'' in \emph{SC '08: Proceedings of the 2008 ACM/IEEE
Conference on Supercomputing}, Austin, TX, USA, 2008, pp.~1-11, doi:
10.1109/SC.2008.5222005.

{[}2{]} S. W. Keckler, W. J. Dally, B. Khailany, M. Garland, and D.
Glasco, ``GPUs and the Future of Parallel Computing,'' in \emph{IEEE
Micro}, vol.~31, no. 5, Sept.-Oct.~2011, pp.~7-17, doi:
10.1109/MM.2011.89.

{[}3{]} S. Shariar and K. M. Azharul Hasan, ``GPU Accelerated Indexing
for High Order Tensors in Google Colab,'' 2020 IEEE Region 10 Symposium
(TENSYMP), Dhaka, Bangladesh, 2020, pp.~686-689, doi:
10.1109/TENSYMP50017.2020.9230789.


% Can use something like this to put references on a page
% by themselves when using endfloat and the captionsoff option.
\ifCLASSOPTIONcaptionsoff
  \newpage
\fi

% trigger a \newpage just before the given reference
% number - used to balance the columns on the last page
% adjust value as needed - may need to be readjusted if
% the document is modified later
%\IEEEtriggeratref{8}
% The "triggered" command can be changed if desired:
%\IEEEtriggercmd{\enlargethispage{-5in}}

% Uncomment when use biblatex with style=ieee
%\renewcommand{\bibfont}{\footnotesize} % for IEEE bibfont size

\pagebreak[3]
% that's all folks
\end{document}

